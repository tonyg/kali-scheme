\chapter{Threads}

% safety (and the lack thereof)

This chapter describes Scheme~48's thread system: Scheme~48 threads
are fully preemptive; all threads (currently) run within a single
operating system process.  Scheme~48 allows writing customized, nested
schedulers, and provides numerous facilities for the synchronization
of shared-memory programs, most importantly \textit{proposals} for
optimistic concurrency.

\section{Creating and controlling threads}

The bindings described in this section are part of the \code{threads}
structure.
%
\begin{protos}
\proto{spawn}{ thunk}{thread}
\proto{spawn}{ thunk name}{thread}
\end{protos}
%
\code{Spawn} creates a new thread, passes that thread to the current
scheduler, and instructs the scheduler to run \cvar{thunk} in that
thread.  The \cvar{name} argument (a symbol) associates a symbolic
name with the thread; it is purely for debugging purposes.
%
\begin{protos}
\protonoresult{relinquish-timeslice}{}
\protonoresult{sleep}{ time-in-milliseconds}
\protonoresult{terminate-current-thread}{}
\end{protos}
%
\code{Relinquish-timeslice} instructs the scheduler to run another
thread, thus relinquishing the timeslice of the current thread.
\code{Sleep} does the same and asks the scheduler to suspend the
current thread for at least \cvar{time-in-milliseconds} milliseconds
before resuming it.  Finally, \code{terminate-current-thread}
terminates the current thread.

Each thread is represented by a thread object.  The following
procedures operate on that object:
%
\begin{protos}
\proto{current-thread}{}{thread}
\proto{thread?}{ thing}{boolean}
\proto{thread-name}{ thread}{name}
\proto{thread-uid}{ thread}{integer}
\end{protos}
%
\code{Current-thread} returns the thread object associated with the
currently running thread.  
\code{Thread?} is the predicate for thread objects.
\code{Thread-name} extracts the name of the thread, if one was
specified in the call to \code{spawn}, \code{\#f} otherwise.
\code{Thread-uid} returns the \textit{uid} of the thread, a unique
integer assigned by the thread system.

\section{Advanced thread handling}

The following bindings are part of the \code{threads-internal} structure:
%
\begin{protos}
\protonoresult{terminate-thread!}{ thread}
\protonoresult{kill-thread!}{ thread}
\end{protos}
%
\code{Terminate-thread!} unwinds the thread
associated with \cvar{thread}, running any pending \code{dynamic-wind}
\cvar{after} thunks (in that thread), after which the thread
terminates.  \code{Kill-thread!} causes the thread associated with
\cvar{thread} to terminate immediately without unwinding its continuation.
%


\section{Debugging multithreaded programs}

Debugging multithreaded programs can be difficult.

As described \link*{here}[in section~\Ref]{cells}, when any thread signals an
 error, Scheme~48 stops running all of the threads at that command level.
 
 The following procedure (exported by the structure
 \code{debug-messages}) is useful in debugging multi-threaded
 programs.
\begin{protos}
\protonoresult{debug-message}{ element$_0$ \ldots}
\end{protos}
\code{Debug-message} prints the elements to `\code{stderr}', followed by a
 newline.
The only types of values that \code{debug-message} prints in full are small
 integers (fixnums), strings, characters, symbols, booleans, and the empty list.
Values of other types are abbreviated as follows:
%
\begin{center}
\begin{tabular}{ll}
 pair       &   \code{(...)}\\
 vector     &   \code{\#(...)}\\
 procedure  &   \code{\#\{procedure\}}\\
 record     &   \code{\#\{<name of record type>\}}\\
 all others &   \code{???}\\
\end{tabular}
\end{center}
%
The great thing about \code{debug-message} is that it bypasses Scheme~48's
 I/O and thread handling.
The message appears immediately, with no delays or errors.

\section{Optimistic concurrency}

%add an overview

A \cvar{proposal} is a record of reads from and and writes to locations in
 memory.  Each thread has an associated \textit{current proposal}
 (which may be \code{\#f}).
The \cvar{logging} operations listed below record any values read or
 written in the current proposal.
A reading operation, such as \code{provisional-vector-ref}, first checks to
 see if the current proposal contains a value for the relevent location.
If so, that value is returned as the result of the read.
If not, the current contents of the location are stored in the proposal and
 then returned as the result of the read.
A logging write to a location stores the new value as the current contents of
 the location in the current proposal; the contents of the location itself
 remain unchanged.

\cvar{Committing} to a proposal verifies that any reads logged in
 the proposal are still valid and, if so, performs any writes that
 the proposal contains.
A logged read is valid if, at the time of the commit, the location contains
 the same value it had at the time of the original read (note that this does
 not mean that no change occured, simply that the value now is the same as
 the value then).
If a proposal has an invalid read then the effort to commit fails; no change
 is made to the value of any location.
The verifications and subsequent writes to memory are performed atomically
 with respect to other proposal commit attempts.
% Explain better.  Add an example?

\begin{protos}
\proto{ensure-atomicity}{ thunk}{value(s)}
\protonoresult{ensure-atomicity!}{ thunk}
\end{protos}
\noindent
If there is a proposal in place 
 \code{ensure-atomicity} and \code{ensure-atomicity!}
 simply make a (tail-recursive) call to \cvar{thunk}.
If the current proposal is \code{\#f} they create a new proposal,
 install it, call \cvar{thunk}, and then try to commit to the proposal.
This process repeats, with a new proposal on each iteration, until
 the commit succeeds.
\code{Ensure-atomicity} returns whatever values are returned by \cvar{thunk}
 on its final invocation, while \code{ensure-atomicity!} discards any such
 values and returns nothing.

\begin{protos}
\proto{provisional-car}{ pair}{value}
\proto{provisional-cdr}{ pair}{value}
\protonoresult{provisional-set-car!}{ pair value}
\protonoresult{provisional-set-cdr!}{ pair value}
\proto{provisional-cell-ref}{ cell}{value}
\protonoresult{provisional-cell-set!}{ cell value}
\proto{provisional-vector-ref}{ vector i}{value}
\protonoresult{provisional-vector-set!}{ vector i value}
\proto{provisional-string-ref}{ vector i}{char}
\protonoresult{provisional-string-set!}{ vector i char}
\proto{provisional-byte-vector-ref}{ vector i}{k}
\protonoresult{provisional-byte-vector-set!}{ vector i k}
\end{protos}
\noindent
These are all logging versions of their Scheme counterparts.
Reads are checked when the current proposal is committed and writes are
 delayed until the commit succeeds.
If the current proposal is \code{\#f} these perform exactly as their Scheme
 counterparts.

The following implementation of a simple counter may not function properly
 when used by multiple threads.
\begin{example}
(define (make-counter)
  (let ((value 0))
    (lambda ()
      (set! value (+ value 1))
      value)))
\end{example}

Here is the same procedure using a proposal to ensure that each
 increment operation happens atomically.
The value of the counter is kept in a
 \link*{cell}[cell (see section~\Ref]{cells}
 to allow the use of
 logging operations.
\begin{example}
(define (make-counter)
  (let ((value (make-cell 0)))
    (lambda ()
      (ensure-atomicity
        (lambda ()
          (let ((v (+ (provisional-cell-ref value)
                      1)))
            (provisional-cell-set! value v)
            v))))))
\end{example}

Because \code{ensure-atomicity} creates a new proposal only if there is
 no existing proposal in place, multiple atomic actions can be merged
 into a single atomic action.
For example, the following procedure increments an arbitrary number of
 counters at the same time.
This works even if the same counter appears multiple times;
 \code{(step-counters! c0 c0)} would add two to the value of counter \code{c0}.
\begin{example}
(define (step-counters! . counters)
  (ensure-atomicity
    (lambda ()
      (for-each (lambda (counter)
                  (counter))
                counters))))
\end{example}

\begin{example}
(define-synchronized-record-type \cvar{tag} \cvar{type-name}
  (\cvar{constructor-name} \cvar{field-tag} \ldots)
  [(\cvar \cvar{field-tag} \ldots)]
  \cvar{predicate-name}
  (\cvar{field-tag} \cvar{accessor-name} [\cvar{modifier-name}])
  \ldots)
\end{example}
This is the same as \code{define-record-type}
 except all field reads and
 writes are logged in the current proposal.
If the optional list of field tags is present then only those fields will
 be logged.

\begin{protos}
\proto{atomically}{ thunk}{value(s)}
\protonoresult{atomically!}{ thunk}
\end{protos}
\noindent
\code{Atomically} and \code{atomically!} are identical
 to \code{ensure-atomicity} and \code{ensure-atomicity!} except that they
 always install a new proposal before calling \code{thunk}.
The current proposal is saved and then restored after \code{thunk} returns.
\code{Atomically} and \code{atomically!} are useful if \code{thunk} contains
 code that is not to be combined with any other operation.
% example?

The following procedures give access to the low-level proposal mechanism.
\begin{protos}
\proto{maybe-commit}{ proposal}{boolean}
\proto{make-proposal}{}{proposal}
\proto{current-proposal}{}{proposal}
\protonoresult{set-current-proposal!}{ proposal}
\proto{with-proposal}{ proposal thunk}{value \ldots}
\syntaxproto{with-new-proposal}{ (\cvar{lose}) \cvar{exp} \ldots}{value \ldots}
\end{protos}
\noindent
\code{Maybe-commit} verifies that any reads logged in \cvar{proposal} are
 still valid and, if so, performs any writes that \cvar{proposal} contains.
A logged read is valid if, at the time of the commit, the location read contains
 the same value it had at the time of the original read (note that this does
 not mean that no change occured, simply that the value now is the same as
 the value then).
\code{Maybe-commit} returns \code{\#t} if the commit succeeds and \code{\#f}
 if it fails.

\code{Make-proposal} creates a new proposal.
\code{Current-proposal} and \code{set-current-proposal} access and set
 the current thread's proposal.
It is an error to pass to \code{set-current-proposal!} a proposal that
 is already in use.

\code{With-proposal} saves the current proposal, installs \cvar{proposal} as
 the current proposal, and then calls \cvar{thunk}.
When \cvar{thunk} returns the saved proposal is reinstalled as the current
proposal
and the value(s) returned by \cvar{thunk} are returned.
\code{With-new-proposal} saves the current proposal, installs a new
one, executes the forms in the body, and returns whatever they
returns.  It also binds \cvar{lose} to a thunk repeating the
procedure of installing a new procedure and running the body.
Typically, the body will call \code{maybe-commit} and, if that fails,
call \cvar{lose} to try again.

\section{Condition variables}
% these require proposals

\textit{Condition variables} (defined in the \code{condvars}
structure) allow threads perform condition synchronization: It allows
threads to block, waiting for a specified condition---associated with a
condition variable---to occur, and other threads to wake up the waiting
threads when the condition is fulfilled.

Note that, in Scheme~48, condition variables work in conjunction with
proposals, not with mutex locks or semaphores, as in most other
implementations of this concept.

\begin{protos}
\proto{make-condvar}{}{condvar}
\proto{make-condvar}{ id}{condvar}
\proto{condvar?}{ thing}{boolean}
\protonoresult{set-condvar-has-value?!}{ condvar boolean}
\proto{condvar-has-value?}{ condvar}{boolean}
\protonoresult{set-condvar-value!}{ condvar value}
\proto{condvar-value}{ condvar}{value}
\proto{maybe-commit-and-wait-for-condvar}{ condvar}{boolean}
\proto{maybe-commit-and-set-condvar!}{ condvar value}{boolean}
\end{protos}
%
\code{Make-condvar} creates a condition variable.  (The optional
\cvar{id} argument is only for debugging purposes; the discloser for
condition variables prints it out if present.)  \code{Condvar?} is the
predicate for condition variables.

Each condition variable has an associated value and a flag
\code{has-value?} signalling if the condition has already occured.
The accessor for flag is \code{condvar-has-value?};
\code{set-condvar-has-value?!} sets it.  Both are provisional
operations and go through the current proposal.
\code{Set-condvar-value!} sets the value of the condition variable
(unprovisionally), and \code{condvar-value} extracts it.

\code{Maybe-commit-and-wait-for-condvar} attempts to commit the
current proposal.  If the commit succeeds, it suspends the current
thread and registers it with the \cvar{condvar} condition variable.
Upon waking up again \code{maybe-commit-and-wait-for-condvar} returns
\code{\#t}, If the commit fails, \code{maybe-commit-and-set-condvar}
returns \code{\#f}.

\code{Maybe-commit-and-set-condvar!} sets the value of the
\cvar{condvar} condition variable to \cvar{value}, (provisionally)
sets the \code{has-value?} flag to \code{\#t}, and then attempt to
commit the current proposal.  Upon success, it wakes up all suspended
threads registered with \cvar{condvar} and returns \code{\#t},
otherwise, it returns \code{\#f}.

\section{Mutual exclusion}

Scheme~48 also has more traditional mutual-exclusion synchronization
abstractions, specifically mutex locks and placeholders.  Note that
typically synchronization via optimistic concurrency is usually
preferable: Mutual exclusion often puts the running program into an
inconsistent state for the time of the inclusion, which has adverse
effects on modularity and interruptibility.

\subsection{Locks}

The \code{locks} structure contains bindings that implement standard
mutex locks:
%
\begin{protos}
\proto{make-lock}{}{lock}
\proto{lock?}{ thing}{boolean}
\protonoresult{obtain-lock}{ lock}
\proto{maybe-obtain-lock}{ lock}{boolean}
\protonoresult{release-lock}{ lock}
\end{protos}
%
\code{Make-lock} creates a lock in the ``released'' state.
\code{Lock?} is the predicate for locks.

\code{Obtain-lock} atomically checks if \cvar{lock} is in the
``released'' state.  If it is, the lock is put into the ``obtained''
state, and \code{obtain-lock} returns immediately.  If the lock is in
the ``obtained'' state, the current thread is suspended and registered
with the lock.
\code{Maybe-obtain-lock}, like \code{obtain-lock}, checks the state of
\cvar{lock}: if it is ``released,'' the lock is put into the
``obtained'' state, if it is ``obtained,'' \code{maybe-obtain-lock}
returns immediately.  \code{Maybe-obtain-lock} returns \code{\#t} if
it was able to obtain the lock, and \code{\#f} otherwise.

\code{Release-lock} does nothing if \cvar{lock} is in the ``released''
state.  If it is in the ``obtained'' state, \code{release-lock}
causes one of the threads suspended on an \code{obtain-lock} lock
operation to continue execution.  If that thread is the last thread
registered with the lock, the lock is transferred to the ``released''
state.  In any case, \code{release-lock} returns immediately.

\subsection{Placeholders}

The \code{placeholders} structure contains bindings for
\textit{placeholders}---thread-safe, write-once variables, akin to
ID-90 I-structures or CML I-variables.

The typical scenario for placeholders is that, say, a thread~A
computes a value needed by another thread~B at some unspecified time.
Both threads share access to a placeholder; when A has computed the
value, it places it into the placeholder.  When B needs the value, it
extracts it from placeholder, blocking if necessary.
%
\begin{protos}
\proto{make-placeholder}{}{placeholder}
\proto{make-placeholder}{ id}{placeholder}
\proto{placeholder?}{ thing}{boolean}
\protonoresult{placeholder-set!}{ placeholder value}
\proto{placeholder-value}{ placeholder}{value}
\end{protos}
%
\code{Make-placeholder} creates an empty placeholder.  (The optional
\cvar{id} argument is only for debugging purposes; the discloser for
placeholders prints it out if present.)  \code{Placeholder?} is the
predicate for placeholders.

\code{Placeholder-set!} places a value into a placeholder.  Doing this
more than once signals an error.  \code{Placeholder-value} extracts
the value from the placeholder and returns it.  If the placeholder is
empty, it blocks the current thread until it becomes full.



\section{Writing custom synchronization abstractions}

The bindings explained in this section are part of the
\code{threads-internal} structure.  They are concerned with suspending
threads and making them runnable again upon some later event.

Typically, a suspended thread needs to be recorded in a queue
somewhere for later waking-up.  To allow a thread to be recorded in
multiple queues (say, when it waits for one of a number of events),
such \textit{thread queues} are ordinary queues containing cells that,
in turn, contain the thread objects themselves.  Each thread has at
most one such cell associated with it which is shared among all queues
(or other data structures) holding on to the suspended thread.  The
cell is cleared when the thread is woken up.
%
\begin{protos}
\proto{thread-queue-empty?}{ thread-queue}{boolean}
\proto{maybe-dequeue-thread!}{ thread-queue}{boolean}
\end{protos}
%
\code{Thread-queue-empty?} atomically checks whether the
\cvar{thread-queue} thread queue is empty, i.e., if it does not
contain non-empty cells.  \code{Maybe-dequeue-thread!} provisionally
dequeues a thread from \cvar{thread-queue} if it contains one.  It
returns the dequeued thread or \code{\#f} if the queue is empty.
%
\begin{protos}
\proto{maybe-commit-and-block}{ cell}{boolean}
\proto{maybe-commit-and-block-on-queue}{ thread-queue}{boolean}
\proto{maybe-commit-and-make-ready}{ thread-or-queue}{boolean}
\end{protos}
%
\code{Maybe-commit-and-block} attempts to commit the current proposal.
If this succeeds, the current thread is blocked, the thread's cell is
set to \cvar{cell}, and \code{\#t} is returned.  Otherwise, \code{\#f}
is returned.  \code{Maybe-commit-and-block-on-queue} is like
\code{maybe-commit-and-block}, excepts that it creates a fresh cell
for the thread and enqueues it in \cvar{thread-queue} if the commit
succeeds.

\code{Maybe-commit-and-make-ready} accepts either a thread object or a
thread queue as an argument.  In either case,
\code{maybe-commit-and-make-ready} tries to commit the current
proposal.  If that succeeds, it \code{maybe-commit-and-make-ready}
makes its argument runnable: if \cvar{thread-or-queue} is a thread,
that thread is made runnable, if it is a thread queue, all threads on
the queue are made runnable.  (In the latter case, none of the threads
actually runs until all have been made runnable.)
\code{Marybe-commit-and-make-ready} returns \code{\#t} if it succeeded,
and \code{\#f} otherwise.

% \section{Writing your own schedulers}


%%% Local Variables: 
%%% mode: latex
%%% TeX-master: "manual"
%%% End: 
